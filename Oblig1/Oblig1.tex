%\UseRawInputEncoding
\documentclass{article}
\setcounter{secnumdepth}{0}
\usepackage[T1]{fontenc}
\usepackage[utf8]{inputenc}
%\usepackage[latin1]{inputenc}
%\usepackage[english, norsk]{babel}
\usepackage{filecontents}
\usepackage{tcolorbox}
\usepackage{url}
\usepackage{etoolbox}
\usepackage{framed}
\usepackage{framed, color}
\usepackage{xcolor}
\usepackage{mdframed}
\usepackage{float}
\usepackage{gensymb}
\usepackage{amsmath}

\definecolor{Black}{rgb}{0.0, 0.0, 0.0}

%Definer kode
\usepackage{listings}
\usepackage{color}
\definecolor{dkgreen}{rgb}{0,0.6,0}
\definecolor{gray}{rgb}{0.5,0.5,0.5}
\definecolor{mauve}{rgb}{0.58,0,0.82}

\lstset{frame=tb,
extendedchars = true,
texcl=true,
  language=C++,
  aboveskip=3mm,
  belowskip=3mm,
  showstringspaces=false,
  columns=flexible,
  basicstyle={\small\ttfamily},
  numbers=none,
  numberstyle=\tiny\color{gray},
  keywordstyle=\color{blue},
  commentstyle=\color{dkgreen},
  stringstyle=\color{mauve},
  breaklines=true,
  breakatwhitespace=true,
  tabsize=3
}

\usepackage[colorlinks]{hyperref}
\hypersetup{citecolor=Black}
\hypersetup{linkcolor=Black}
\hypersetup{urlcolor=Black}
\usepackage{cleveref}


\setlength{\parindent}{0em}
\setlength{\parskip}{1em}
%\renewcommand{\baselinestretch}{2.0}

%\renewcommand\thesubsection{\alph{subsection}}

\renewcommand{\figurename}{Figure}
\begin{document}
\author{Kent Odde}
\title{Introduction to Cybersecurity, Assignment 1}

\maketitle
\thispagestyle{empty}
\begin{center}
\includegraphics[width=\linewidth,height=0.2\textheight,keepaspectratio]{img/USN.png}
\end{center}
\newpage

\tableofcontents

\newpage

\section{Abstract}

This is my submission for the first assignment in DCS3101, Introduction to Cybersecurity. 

The complete source code for both the Vignere cipher and Enigma program can be found both in the appendices in the appendices of this report, as well as in the folder \textit{src}. Some test vectors for the enigma program have been removed from the report, as they are 70 000 characters long each. 

%Innholdsfortegnelse
\section{Q1}

\textbf{CIA - Confidentiality, Integrity and Availability.}

Alice wants to send the message "Introduction to Data and Cyber-Security (DCS3101)" to Bob. If she wants to employ the CIA concepts in the sending of this message, she has to ensure three things:
\begin{itemize}
\item{\textbf{Confidentiality}: She will have to make sure that Bob is the only one that will be able to read the contents, and that all other third parties will not. Encrypting the message with a block encryption scheme like AES or DES, and using something like cipher-block-chaining, electronic code book etc. to handle the blocks are ways in which Alice may obtain confidentiality.}
\item{\textbf{Integrity}: This means making sure that Bob can be certain that the message he receives is in fact the intended message communicated by Alice. Simply put, he needs to know that the contents has not been altered in any way. Integrity can be provided by using a hashing function as a signature. If Bob can match up the provided signature with the output of hashing part of the message, he can be sure that the message sent from Alice has not been changed in any way.}
\item{\textbf{Availability}: All the security in the world is meaningless if Bob is not able to decrypt the message. The correct contents of the message must be available for the intended receiver, at the right time. This explanation is a bit superficial, as this topic haven't been covered as much as the others so far in the course. }
\end{itemize}

These three properties are by no means independent, and one will always have to analyze ones needs, and do a trade-off analysis between the three.

High confidentiality and integrity may lead to lowered availability. Either in terms of time it takes to decrypt the message or the complexity of it. In the extreme case, the confidentiality and integrity is so high that the availability is equal to zero. On the other hand a high emphasis on availability may compromise the level of confidentiality and integrity. 

%Oppgaven
\section{Q2}
\subsection{Illustration of Vignere Ciphering}

In order to explain Vignere ciphering, we have to realize that Vignere ciphering is nothing more than an an expanded and enhanced version of Ceasar ciphering. Let's start off by explaining a Ceasar Cipher. 

\subsubsection{Ceasar Cipher}

Ceasar Cipher is a ciphering scheme where we apply an operation to all of the characters in a string. This may be shifting each character by +5 places. In this case we say that the key is F, as it is the letter with the value of 5, given that we count A as 0.

\begin{figure}[H]
 \centering
  \includegraphics[width=200pt]{img/ceasar.png}
 \caption{Simple Ceasar cipher example}
 \end{figure}

There are many flaws in this scheme, where the most obvious one is that the operation is static, and if you crack one character, you crack all of them. A statistical attack can be very effective here.

\begin{figure}[H]
 \centering
  \includegraphics[width=300pt]{img/frequency.jpg}
 \caption{English letter frequency\cite{LETTER}}
 \end{figure}

In this graph we see that the most common letters in the English language are E, T, A and O. If we assume this is also true for our cipher text in figure 1, \textit{HTIJBTWI}, we can guess that either the T or the I will map to one of these letters. If we start at the left of the graph, and assume that the letter \textit{T} is really \textit{E}, we will only need to test 3 inverse operations (E, T, A), before arriving at the correct one (that I is equal to O). Of course testing for T, can be omitted completely.

 This is a very small sample size, and as the text grows, the statistical properties of the language will be even more likely to correspond with our text. 

\subsubsection{Vignere Cipher}

A vignere cipher attempts to increase the complexity of the Ceasar Cipher by having the operations change for each character. 

Instead of having a key like F in the caesar cipher, we use a key where the number of characters is greater than one. The length of the key naturally increases the robustness of the encryption. 

\begin{figure}[H]
 \centering
  \includegraphics[width=300pt]{img/vigneredraw1.png}
 \caption{Vignere 1}
 \end{figure}

In the example where the key is 'KEY', we use the key 'K' on the first character of the plaintext string, 'E' is the key off the second, and so on. When we reach the last character of the Key, we start over, so that the 4th character of our plaintext will also be encrypted with the key 'K'.

\begin{figure}[H]
 \centering
  \includegraphics[width=300pt]{img/vigneredraw2.png}
 \caption{Vignere 2}
 \end{figure}

In the following figure, we can see that the two O's and the two D's no longer map to the same letter, as was the problem with the Ceasar Cipher. In the ciphertext we can also see a couple of examples where two equal letters, like M and B, map back two different letters in the plaintext. 

\begin{figure}[H]
 \centering
  \includegraphics[width=300pt]{img/vigneredraw3.png}
 \caption{Vignere 3}
 \end{figure}



If the message is shorter than the key, or if the key is not a multiple of the message (like in the example), we may add padding to the plaintext before encrypting. This is a mechanism of making the messages harder to decipher, because the length of the ciphertext, does not necessarily match the length of the plaintext. The padding should not however consist of the same character. If one tries to shift the last few characters through the alphabet, the key may become obvious from this. As an illustration, let us imagine padding the message with the letter A. 

\begin{figure}[H]
 \centering
  \includegraphics[width=300pt]{img/vigneredraw4.png}
 \caption{A very unfortunate padding scheme}
 \end{figure}

This is of course an extreme example, because A will just give the key in plaintext, but no matter which letter we replace A with, the key will reveal itself after a maximum of 26 shift operations. Because of this, the padding scheme should include either a repetition of some part of the message, or random characters. 

\subsection{Vignere Cipher Program}

When writing the Vignere ciphering program, i assumed the following:
\begin{itemize}
\item{Upper case characters will be encrypted within the set of upper case characters}
\item{Lower case characters will be encrypted within the set of lower case characters}
\item{Spaces, commas and periods will not be encrypted and only represent themselves}
\item{Keys are of course case insensitive}
\item{In the part with two keys, I assumed that the purpose is to encrypt the message twice, using two different keys}
\end{itemize}

In the code, I found it easiest to make use of the ascii table, where upper case letters are shifted within the range of 65 and 90, and the lower case letters are in the range of 97 to 122.

\begin{figure}[H]
 \centering
  \includegraphics[width=300pt]{img/ASCII.png}
 \caption{ASCII table \cite{ASCII}}
 \end{figure}

The essence of the code lies in these two functions. We loop through the plaintext, while keeping another integer variable to keep track of where in the keyword we are. We shift the currenct character in the plaintext by the relative value of the current letter in the keyword. 

\begin{lstlisting}
char shiftChar(char c,  char shiftBy, bool backward = false)
{	
	/*
	Checking if the characters upper or lower case, and chooses the asciiboundaries of the respective set.
	*/
	int startAscii = (c >= STARTASCIILOWER) ? STARTASCIILOWER : STARTASCIIUPPER;
	int stopAscii = (c >= STARTASCIILOWER) ? STOPASCIILOWER : STOPASCIIUPPER;

	/*
	Move the character downto the 0-25 range, performing the shift, and moving it back up the correct asciirange. 
	*/
	char output = c - startAscii;
	if(backward)
	{
		output = modulo((output - (shiftBy - STARTASCIILOWER)), (stopAscii - startAscii + 1)) + startAscii;
	}
	else
	{
		output = modulo((output + (shiftBy - STARTASCIILOWER)), (stopAscii - startAscii + 1)) + startAscii;
	}
	return output;
}

std::string encrypt(std::string plainText, std::string key)
{
	int keyIndex = 0;
	std::string cipherText;
	for(int i = 0; i < plainText.size(); i++)
	{
		/*
		Dont encrypt spaces, commas and periods.
		*/
		if(plainText[i] == ' ' || plainText[i] == '.' || plainText[i] == ',')
		{
			cipherText += plainText[i];
			continue;
		}
		cipherText += (shiftChar(plainText[i], key.at(keyIndex)));
		keyIndex = (keyIndex + 1) % key.size();
	}
	return cipherText;
}
\end{lstlisting}

\subsubsection{Single Key}

The text given in the task, encrypted with the key \textit{kongsberg}, became:
\begin{align*}
Dvr\quad wmjgb\quad hbcjt\quad xpb\quad aawdf\quad unfv\quad kno\quad znfq\quad esx.
\end{align*}

The result can also be seen in the figure below:

\begin{figure}[H]
 \centering
  \includegraphics[width=300pt]{img/vignere1key.png}
 \caption{Vignere cipher, 1 key}
 \end{figure}

\subsubsection{Two Keys}


Encrypted twice, with the keys \textit{norway} and \textit{oslo} respectively, became:
\begin{align*}
Ung\quad aiyam\quad gfzio\quad lqh\quad xkkrx\quad cgqs\quad zjo\quad zqxa\quad icr.
\end{align*}

The result can also be seen in the figure below:
\begin{figure}[H]
 \centering
  \includegraphics[width=300pt]{img/vignere2keys.png}
 \caption{Vignere cipher, 2 keys}
 \end{figure}

%Implementation
\section{Q3}
Lets start by defining the terms diffusion and confusion.

Confusion, in essence deals with obscuring the link between the key and the ciphertext, while diffusion deals with hiding the relationship between the plaintext and the ciphertext. More concretely this means that if we change a bit in the key, we must provide enough confusion, so that almost all of the ciphertext will also be affected by this change. In a similar manor, if we change a single bit of the plaintext, diffusion should be provided so that this will change at least half of the bits in the plaintext. 

In practice these two properties can be obtained in different ways. We will have a look at how they are obtained in DES (Data Encryption Standard). 

DES is built of a SP-network, more specifically a feistel structure. It uses substitution and permutation over 16 rounds along with a 56 bit key, which is expanded into 16 round keys. 

Confusion within DES is obtained by the substitution and the diffusion is obtained by the permutation. 

Substitution is basically a rule set where in the case of DES, a six bit input, will be replaced by a four bit outbit, according the the specific S-box.
Permutation changes the bit position of the input, which hides the statistical properties of the original plaintext. 

In DES confusion is also obtained by the expansion function. It takes a 32 bit input, and produces a 48 bit output, by duplicating some of the original bits. 


%Konklusjon
\section{Q4 - Enigma}

In this task, I will assume knowledge about the Enigma machine on the part of the reader, and not go into the details as this would lead to a very extensive task. There are also some conflicting sources of information about the Enigma. I have gathered all data about models, rotors and reflectors from \cite{ENIGMA1} and \cite{ENIGMA2}

The Enigma I, had a scheme where one would choose three rotors out of five, and one reflector out of three. This can be seen in the figure below. 

\textit{Some sources say that the Enigma I had only three rotors and some say five. However I think it was upgraded at some point, so both statements may have been true at some point. The three out of five scheme seems to be the more common one.}

\begin{figure}[H]
 \centering
  \includegraphics[width=300pt]{img/enigmaIspecs.png}
 \caption{Rotor specifications for Enigma I \cite{ENIGMA2}}
 \end{figure}

To increase the complexity and challenge of this task, I decided implementing an M3 Enigma. This model had a total of eight rotors to choose from, instead of five. However, reflector-A seemed to be removed. In order for this program to be compatible with the Enigma I, I have decided to include all three reflectors. 

\begin{figure}[H]
 \centering
  \includegraphics[width=300pt]{img/enigmaM3specs.png}
 \caption{Rotor specifications for Enigma M3 \cite{ENIGMA2}}
 \end{figure}

For the most part this is quite a trivial task. When pushing a button, the rotors will increment. The signal will run from the button, through the plug-board, through the rotors, into the reflector, back through the rotors, back through the plug-board, and deliver an output. In code this will for the most part consist of looking up characters in constant arrays, taking into account the current position of the rotors. 

However the thing that stomped me for a while, were the ring-settings. The ring-settings are not dynamically changed, but are set by twisting the actual rotor relative to itself. This combined with the rotor position made me resort to pen and paper. Finally i was able to come up with a working function for a transformation through a rotor with a certain position and ring setting, which can be seen in the code below:

\begin{lstlisting}
char Rotor::getTransformedChar(char c)
{
//Operation: c = c + offset - ringsetting
	c = intToAsciiChar(modulo(charToAlphabetIndex(c) + offset - ringSetting, alphabet.size()));

//Operation: Normal transformation
	char o = alphabet.at(charToAlphabetIndex(c));

//Operation: o = o + ringSetting;
	o = intToAsciiChar(modulo(charToAlphabetIndex(o) + ringSetting, alphabet.size()));

//Operation: o = o - offset;
	o = intToAsciiChar((modulo(charToAlphabetIndex(o) - offset, alphabet.size())));
	return o;
}
\end{lstlisting}



In order to test the program properly, I generated correct ciphertexts with several different settings at \url{https://cryptii.com/pipes/enigma-machine}.
\begin{figure}[H]
 \centering
  \includegraphics[width=300pt]{img/enigmaTest.png}
 \caption{Test of Enigma}
 \end{figure}
As can be seen in the figure, these strings were quite long in order to go through all the positions several times. These tests will be excluded from the code in the appendix, because of their size.


In the figure below, one can see a string be encrypted, decrypted and a check performed to see that it in fact has returned the correct plaintext back.
\begin{figure}[H]
 \centering
  \includegraphics[width=300pt]{img/enigmaMan.png}
 \caption{Enigma}
 \end{figure}




%Vedlegg
\section{Appendices}
\subsection{Vignere Cipher Code}
\begin{lstlisting}
#include <iostream>
#include <string>
#include <algorithm>


int STARTASCIILOWER = 97;
int STOPASCIILOWER = 122;
int STARTASCIIUPPER = 65;
int STOPASCIIUPPER = 90;

/*
As the %-operator does not behave like modulo on negative numbers, this function is necessary
*/
int modulo(int a, int b)
{
	return (a % b + b) % b;
}

char shiftChar(char c,  char shiftBy, bool backward = false)
{	
	/*
	Checking if the characters upper or lower case, and chooses the asciiboundaries of the respective set.
	*/
	int startAscii = (c >= STARTASCIILOWER) ? STARTASCIILOWER : STARTASCIIUPPER;
	int stopAscii = (c >= STARTASCIILOWER) ? STOPASCIILOWER : STOPASCIIUPPER;

	/*
	Move the character downto the 0-25 range, performing the shift, and moving it back up the correct asciirange. 
	*/
	char output = c - startAscii;
	if(backward)
	{
		output = modulo((output - (shiftBy - STARTASCIILOWER)), (stopAscii - startAscii + 1)) + startAscii;
	}
	else
	{
		output = modulo((output + (shiftBy - STARTASCIILOWER)), (stopAscii - startAscii + 1)) + startAscii;
	}
	return output;
}

std::string encrypt(std::string plainText, std::string key)
{
	int keyIndex = 0;
	std::string cipherText;
	for(int i = 0; i < plainText.size(); i++)
	{
		/*
		Dont encrypt spaces, commas and periods.
		*/
		if(plainText[i] == ' ' || plainText[i] == '.' || plainText[i] == ',')
		{
			cipherText += plainText[i];
			continue;
		}
		cipherText += (shiftChar(plainText[i], key.at(keyIndex)));
		keyIndex = (keyIndex + 1) % key.size();
	}
	return cipherText;
}

std::string decrypt(std::string cipherText, std::string key)
{
	int keyIndex = 0;
	std::string plainText;
	for(int i = 0; i < cipherText.size(); i++)
	{
		/*
		Dont decrypt spaces, commas and periods.
		*/
		if(cipherText[i] == ' ' || cipherText[i] == '.' || cipherText[i] == ',')
		{
			plainText += cipherText[i];
			continue;
		}

		plainText += (shiftChar(cipherText[i], key.at(keyIndex), true));
		keyIndex = (keyIndex + 1) % key.size();
	}
	return plainText;
}


int main(int argc, char* argv[]) 
{
	/*
	Mostly commandline argument handling
	*/
	if(argc < 4)
	{
		std::cout << "Wrong number of arguments passed\n";
		std::cout << "Arguments: \n\t./Vignere cipher -e/d <string> <key> <optionalSecondKey>\n";
		return -1;
	}

	std::string input = argv[2];
	std::string key = argv[3];
	std::transform(key.begin(), key.end(), key.begin(), ::tolower);
	std::string key2 = "";
	if(argc > 4)
	{	
		key2 = argv[4];
		std::transform(key2.begin(), key2.end(), key2.begin(), ::tolower);
	}
	std::string output;

	if(std::string(argv[1]) == "-e")
	{
		output = encrypt(input, key);

		if(key2.compare("") != 0)
		{
			output = encrypt(output, key2); 
		}
	}
	else if(std::string(argv[1]) == "-d")
	{
		output = decrypt(input, key);

		if(key2.compare("") != 0)
		{
			output = decrypt(output, key2);
		}
	}
	else
	{
		std::cout << "Error, expected arguments:\n\t/Vignere cipher -e/d <string> <key> <optionalSecondKey>\n";

		return -1;
	}
	std::cout << "\nInput:\t\t" << input << "\n";
	std::cout << "Key:\t\t" << key << "\n";
	if(key2.compare("") != 0) std::cout << "Key2:\t\t" << key2 << "\n";
	std::cout << "Output:\t\t" << output << "\n"; 
	return 0;
}
\end{lstlisting}

\subsection{Enigma Code}
\subsubsection{main.cpp}
\begin{lstlisting}
#include <string>
#include <iostream>
#include "Rotor.h"
#include "Enigma.h"

std::string formatString(std::string s);

int main(int argc, const char* argv[])
{
	//Setup Enigma
		//Choose rotors 1-8 - Left to right;
		int rotorIds[3] = {4,5,6};
		//Choose Reflektor 1-3 (ABC)
		int reflektorId = 2;
		//Set offset and ringSetting 1-26 - Left to right;
		int offset[3] = {24,12,26};
		int ringSetting[3] = {13,2,21};
		//Set upto ten pairs in the plugboard. A letter may only be connected once
		//The plugboard may also be empty
		std::vector<std::string> plugboardSettings = 
		{"AH", "MB", "TO", "EI", "PL", "KC", "NQ", "ZX", "WR"};
		Enigma enigma(rotorIds, reflektorId, plugboardSettings);
		enigma.setOffset(offset);
		enigma.setRingSetting(ringSetting);
		enigma.printEnigmaStatus();

	
	//Encrypt and decrypt
		std::string input = "ThisCourseIsAnIntroductionToCyberSecurityThisCourseIsAnIntroductionToCyberSecurityThisCourseIsAnIntroductionToCyberSecurity";
		std::transform(input.begin(), input.end(), input.begin(), ::toupper);
		std::string encrypted = enigma.transform(input);
		//Print settings after encryption:
		//enigma.printEnigmaStatus();
		//Set offset back, in order to decrypt.
		enigma.setOffset(offset);
		std::string decrypted = enigma.transform(encrypted);

	//Print results
		std::cout << "Input:     " << formatString(input) << "\n";
		std::cout << "Encrypted: " << formatString(encrypted) << "\n";
		std::cout << "Decrypted: " << formatString(decrypted) << "\n";
		std::cout << "Checking correct decryption of " << decrypted.size() << " characters..\n";

		if(input.compare(decrypted) == 0)
		{
			std::cout << "Decrypt Success\n";
		}
		else
		{
			std::cout << "Decrypt Failure\n";
		}
}

std::string formatString(std::string s)
{
	int num = 5;
	char sep = ' ';
	for(auto it = s.begin(); (num+1) <= std::distance(it, s.end()); it++)
	{
		std::advance(it, num);
		it = s.insert(it, sep);
	}
	return s;
}

\end{lstlisting}
\subsubsection{Enigma Class}
\textbf{Enigma.h}
\begin{lstlisting}
#ifndef ENIGMA_H
#define ENIGMA_H
#include "Rotor.h"
#include <vector>
#include <algorithm>

class Enigma
{
public:
	Enigma(int rotorIds[3], int reflectorId, std::vector<std::string> _plugboardSettings);
	~Enigma();
	void printEnigmaStatus();
	void setRotors(int r[3]);
	void setOffset(int o[3]);
	void setRingSetting(int s[3]);
	std::string transform(std::string& input);
	
private:
	Rotor reflector;
	std::vector<Rotor> rotors;
	std::vector<std::string> plugboardSettings;
	void rotateRotors();
	char charThroughPlugboard(char c);
};
#endif
\end{lstlisting}

\textbf{Enigma.cpp}
\begin{lstlisting}
#include "Enigma.h"

Enigma::Enigma(int rotorIds[3], int reflectorId, std::vector<std::string> _plugboardSettings): reflector(reflectorId, 'r')
{
	Rotor rotor1(rotorIds[0], 'L');
	Rotor rotor2(rotorIds[1], 'M');
	Rotor rotor3(rotorIds[2], 'R');
	rotors.push_back(rotor1);
	rotors.push_back(rotor2);
	rotors.push_back(rotor3);
	plugboardSettings = _plugboardSettings;
}

Enigma::~Enigma()
{

}

void Enigma::printEnigmaStatus()
{
	std::cout << "Enigma Status: \n\tPlugboard: \n\t";
	for(auto pair : plugboardSettings)
	{
		std::cout << pair << " ";
	} 

	std::cout << "\n\nRotor information from left to right: \n\n";

	for(auto rotor : rotors)
	{
		rotor.printRotorStatus();
	} 

	reflector.printRotorStatus();
}

void Enigma::setRotors(int r[3])
{

	for(int i = 0; i <= 2; i++)
	{
		Rotor rotor(r[i], r[i]);
		rotors.push_back(rotor);
	}
}

void Enigma::setOffset(int o[3])
{
	for(int i = 0; i < rotors.size(); i++)
	{
		rotors.at(i).setOffset(o[i] - 1);
	}

}

void Enigma::setRingSetting(int s[3])
{
	for(int i = 0; i < rotors.size(); i++)
	{
		rotors.at(i).setRingSetting(s[i] - 1);
	}

}

void Enigma::rotateRotors()
{	
	for(auto it = rotors.rbegin(); it != rotors.rend(); it++)
	{
		bool next = it->incrementOffset(rotors.at(1).isDoubleStep());
		if(!next) break;
	}	
	//printRotorStatus();
	
}

std::string Enigma::transform(std::string& input)
{
	std::string output;
	
	for(const char c : input)
	{
		rotateRotors();

		char temp = std::toupper(c);
		//Through the Plugboard
		temp = charThroughPlugboard(temp);

		//Through the rotors
		for(auto it = rotors.rbegin(); it != rotors.rend(); it++)
		{
			temp = it->getTransformedChar(temp);
		}
		//Through the reflector
		temp = reflector.getTransformedChar(temp);
		
		//Through the rotors backwards
		for(auto it = rotors.begin(); it != rotors.end(); it++)
		{
			temp = it->getInverseTransformedChar(temp);
		}

		temp = charThroughPlugboard(temp);

		output += temp;
	}
	return output;
}

char Enigma::charThroughPlugboard(char c)
{
	for(auto pair : plugboardSettings)
	{
		if(pair.front() == c) return pair.back();
		if(pair.back() == c) return pair.front();
	}
	return c;
}
\end{lstlisting}
\subsubsection{Rotor Class}
\textbf{Rotor.h}
\begin{lstlisting}
#include <string>
#include <iostream>
#ifndef ROTOR_H
#define ROTOR_H

class Rotor
{
public:
	Rotor();
	Rotor(int _id, char _pos);
	~Rotor();
	char getTransformedChar(char c);
	char getInverseTransformedChar(char c);
	void setRingSetting(int i);
	bool incrementOffset(bool doubleStep);
	void setOffset(int i);
	void printRotorStatus();
	bool isDoubleStep();
private:
	std::string alphabet;
	int ringSetting;
	int offset;
	int steppingPoint;
	int steppingPoint2 = -1;
	char intToAsciiChar(int i);
	int rotorId = 0;
	char rotorPosition;
	int charToAlphabetIndex(char c);
	int modulo(const int& a, const int& b);
	const int asciiOffset = 65;
};

//Assorted rotor data etc. 
static const std::string masterAlphabet = "ABCDEFGHIJKLMNOPQRSTUVWXYZ";
static const std::string rotorAlphabets[9] = 
{
	"", //Dummy alphabet
	"EKMFLGDQVZNTOWYHXUSPAIBRCJ",
	"AJDKSIRUXBLHWTMCQGZNPYFVOE",
	"BDFHJLCPRTXVZNYEIWGAKMUSQO",
	"ESOVPZJAYQUIRHXLNFTGKDCMWB",
	"VZBRGITYUPSDNHLXAWMJQOFECK",
	"JPGVOUMFYQBENHZRDKASXLICTW",
	"NZJHGRCXMYSWBOUFAIVLPEKQDT",
	"FKQHTLXOCBJSPDZRAMEWNIUYGV"
};
//Rotors will step the next, when stepping to these characters
static const std::string stepPoints[] = 
{
	"0", //Dummy stepper
	"R",
	"F",
	"W",
	"K",
	"A",
	"AN" //Have to add another stepping point at 'N' for this
};

static const std::string reflectorAlphabets[] = 
{
	"", //Dummy alphabet
	"EJMZALYXVBWFCRQUONTSPIKHGD",
	"YRUHQSLDPXNGOKMIEBFZCWVJAT",
	"FVPJIAOYEDRZXWGCTKUQSBNMHL"
};
#endif
\end{lstlisting}

\textbf{Rotor.cpp}
\begin{lstlisting}
#include "Rotor.h"

Rotor::Rotor()
{

}

Rotor::Rotor(int _id, char _pos)
{
	//A bit of extra overhead here, because a rotor may be a reflektor, or it may have two steppingpoints

	rotorPosition = _pos;
	if(_pos == 'r')
	{
		rotorId = _id;
		alphabet = reflectorAlphabets[_id];
	}
	else
	{
		rotorId = _id;
		alphabet = rotorAlphabets[_id];
	}
	ringSetting = 0;
	offset = 0;
	
	if(_id == 6 || _id == 7 || _id == 8)
	{
		steppingPoint = charToAlphabetIndex(stepPoints[6].at(0));
		steppingPoint2 = charToAlphabetIndex(stepPoints[6].at(1));
	}
	else
	{
		steppingPoint = charToAlphabetIndex(stepPoints[_id].at(0));
	}
}

Rotor::~Rotor()
{

}

int Rotor::modulo(const int& a, const int& b)
{
	return (b + (a % b)) % b;
}

char Rotor::getTransformedChar(char c)
{
//Operation: c = c + offset - ringsetting
	c = intToAsciiChar(modulo(charToAlphabetIndex(c) + offset - ringSetting, alphabet.size()));

//Operation: Normal transformation
	char o = alphabet.at(charToAlphabetIndex(c));

//Operation: o = o + ringSetting;
	o = intToAsciiChar(modulo(charToAlphabetIndex(o) + ringSetting, alphabet.size()));

//Operation: o = o - offset;
	o = intToAsciiChar((modulo(charToAlphabetIndex(o) - offset, alphabet.size())));
	return o;
}

char Rotor::getInverseTransformedChar(char c)
{	
//Operation: c = c + offset;
	c = intToAsciiChar(modulo(charToAlphabetIndex(c)+offset, alphabet.size()));

//Operation: c = c -ringSetting;
	c = intToAsciiChar(modulo(charToAlphabetIndex(c) - ringSetting, alphabet.size()));

//Operation: What gives c in masteralphabet:
	int i = alphabet.find(c);
	char o = masterAlphabet.at(i);

//Operation: o = o - offset + ring
	o = intToAsciiChar(modulo(charToAlphabetIndex(o) - offset + ringSetting, alphabet.size()));

	return o;
}

void Rotor::setRingSetting(int i)
{
 	ringSetting = i;
}

bool Rotor::incrementOffset(bool doubleStep)
{
	offset = modulo(offset + 1, alphabet.size());
	//If this is the rightmost rotor, and the middle rotor is in doublestep position, we also have to return true;
	if((offset == steppingPoint || offset == steppingPoint2) || rotorPosition == 'R' && doubleStep) 
	{
		return true;
	}
	return false;
}

void Rotor::setOffset(int i)
{
	offset = i;
}

char Rotor::intToAsciiChar(int i)
{
	return char(i + asciiOffset);
}

int Rotor::charToAlphabetIndex(char c)
{
	int i = (int)c - asciiOffset;
	return i;
}

void Rotor::printRotorStatus()
{
	if(rotorPosition == 'r')
	{
		std::cout << "\tReflector: "<< rotorId << "\n";
		std::cout << "\t\tAlphabet: " << alphabet << "\n\n\n";
	}
	else
	{
		std::cout << "\t" << rotorPosition << "-rotor\n";
		std::cout << "\t\tRotorId: " << rotorId << "\n";
		std::cout << "\t\tStepping point: " << steppingPoint << "\n";
		if(steppingPoint2 != -1) std::cout << "\t\tStepping point2: " << steppingPoint2 << "\n";
		std::cout << "\t\tCurrent offset:" << offset << "\n";
		std::cout << "\t\tCurrent RingSetting: " << ringSetting << "\n";
		std::cout << "\t\tAlphabet: " << alphabet << "\n\n\n";
	}
}
//If the middle rotor is at the position before it is supposed to step the next, it shall just step to the next;
bool Rotor::isDoubleStep()
{
	bool ret = (rotorPosition == 'M' && (offset == modulo(steppingPoint - 1, alphabet.size()) || (steppingPoint2 != -1 && offset == modulo(steppingPoint2 -1, alphabet.size()))));
	return ret;
}
\end{lstlisting}

\newpage
%Referanse
%\section{Referanser}

\nocite{*}
\bibliographystyle{plain}
\bibliography{ref}

\addcontentsline{toc}{section}{References}

\end{document}